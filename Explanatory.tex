\documentclass{article}

\newcommand{\SAT}{\textnormal{$3$-SAT}}

\title{Full Derandomization of Sch\"{o}ning's $\SAT$ Algorithm}
\author{}
\date{}

\begin{document}

\maketitle

\section{Introduction}
\paragraph{} In section 2, we present Sch\"{o}ning's original randomized algorithm\cite{Schoning99} for $\SAT$. In section 3, a derandomization algorithm by \cite{Dantsin02} is given. In section 4, we present an improvement by \cite{Moser11} that builds on the derandomization algorithm whose performance is arbitrarily close to the randomized version, thereby claiming a full derandomization.

\section{Sch\"{o}ning's $3$-SAT Algorithm\cite{Schoning99}}



\section{A Derandomization}

\section{Improving on the Derandomization}

\begin{thebibliography}{9}
\bibitem{Dantsin02}
E. Dantsin, A. Goerdt, E. A. Hirsch, R. Kannan, J. Kleinberg, C. Papadimitriou, O. Raghavan, and U. Sch\"{o}ning. 
\textit{A deterministic $(2-2/(k+1))n$ algorithm for k-SAT based on local search.} Theoretical Computer Science 289, (2002).

\bibitem{Moser11}
R. Moser and D. Scheder.
\textit{A Full Derandomization of Sch\"{o}ning’s k-SAT Algorithm.} Proceedings of the 44th Annual ACM Symposium on Theory of Computing, (2011).

\bibitem{Schoning99}
U. Sch\"{o}ning.
\textit{A Probabilistic Algorithm for $k$-SAT and Constraint Satisfaction Problems.} Proceedings of the 40th Annual Symposium on Foundations of Computer Science, (1999).

\end{thebibliography}

\end{document}

