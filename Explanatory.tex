\documentclass{article}

\usepackage{amsthm}
\usepackage[ruled,vlined,linesnumbered]{algorithm2e}

\newcommand{\SAT}{\textnormal{$3$-SAT}}
\newcommand{\CNF}{\textnormal{$3$-CNF}}
\newcommand{\vbl}[1]{\textnormal{vbl(#1)}}
\newcommand{\dist}[2]{d_H(#1,#2)}
\newcommand{\ball}[2]{B_{#1}(#2)}

\newtheorem{definition}{Definition}


\title{Full Derandomization of Sch\"{o}ning's $\SAT$ Algorithm}
\author{}
\date{}

\begin{document}

\maketitle

\section{Introduction}
\paragraph{} In section 2, we introduce some notations and definitions used in this paper. In section 3, we present Sch\"{o}ning's original randomized algorithm\cite{Schoning99} for $\SAT$. In section 4, a derandomization algorithm based on local search by \cite{Dantsin02} is given. In section 5, we present an improvement by \cite{Moser11} that builds on the derandomization algorithm whose performance is arbitrarily close to the randomized version, thereby claiming a full derandomization.

\section{Preliminaries}
\paragraph{} An input instance of a $\SAT$ problem is a $\CNF$(Conjunctive Normal Form) formula $F$ with $n$ variables. $F$ contains a finite set of clauses where each clause $C$ contains at most $3$ pairwise independent literals. A literal $u$ is either a variable $x$ or a complemented variable $\bar{x}$. Let $V$ denote the set of variables in $F$. \par 
A truth assignment $\alpha: V\rightarrow \{0,1\}$ assigns each variable a boolean value. If a literal $u = x$, it is satisfied by $\alpha$ iff $\alpha(x) = 1$. If $u = \bar{x}$, it is satisfied by $\alpha$ iff $\alpha(x) = 0$. A clause $C$ is satisfied by $\alpha$ if at least one of its literals are satisfied by $\alpha$. A formula $F$ is satisfied by $\alpha$ if all clauses are satisfied by $\alpha$. \par 
Given two assignment $\alpha$ and $\beta$, the hamming distance between them $\dist{\alpha}{\beta}$ is defined to be the number of variables $x \in V$ where $\alpha(x) \neq \beta(x)$. A Hamming Ball centered at $\alpha$ with radius $r$ is defined to be $\ball{r}{\alpha} = \{\beta: \dist{\alpha}{\beta} \leq r\}$ that is, all assignments whose hamming distance from $\alpha$ is less or equals to $r$. \par 
\begin{definition}[Promise-Ball-\SAT\cite{Moser11}]
	Given a $\CNF$ formula $F$, an assignment $\alpha$ for $F$, a natural number $r$ and the promise that the Hamming Ball $\ball{r}{\alpha}$ contains a satisfying assignment for $F$. Find any satisfying assignment.
\end{definition}
The Promise-Ball-$\SAT$ problem is an important sub-problem in solving $\SAT$. 

\section{Sch\"{o}ning's $3$-SAT Algorithm\cite{Schoning99}}
The randomized algorithm for $\SAT$ builds on a simple random walk algorithm for Promise-Ball-$\SAT$. \par 
\begin{algorithm}[H]
	\caption{Promise-Ball-$\SAT$-Random-Walk}
	\DontPrintSemicolon
	\SetKwInOut{Input}{Input}\SetKwInOut{Output}{Output}
	\Input{$\CNF$ formula $F$ with $n$ variables}
	\Output{A satisfying assignment $\alpha^*$ or NO if $\alpha^*$ is not found}
	\BlankLine
	Initialization: Choose a assignment $\alpha$ for $F$ uniformly at random\\
	\For{steps = 1 to 3n}{
		\lIf{$\alpha$ satisfies $F$}{return $\alpha$}
		Choose an arbitrary clause $C$ not satified by $\alpha$\\
		Choose a literal in $C$ uniformly at random and flip its value in $\alpha$
	}
	\lIf{$\alpha$ satisfies $F$}{return $\alpha$}
	\hspace{7em}\lElse{return NO}
\end{algorithm}

\section{Derandomization with Local Search\cite{Dantsin02}}

\section{Improving on the Derandomization\cite{Moser11}}

\begin{thebibliography}{9}
\bibitem{Dantsin02}
E. Dantsin, A. Goerdt, E. A. Hirsch, R. Kannan, J. Kleinberg, C. Papadimitriou, O. Raghavan, and U. Sch\"{o}ning. 
\textit{A deterministic $(2-2/(k+1))n$ algorithm for k-SAT based on local search.} Theoretical Computer Science 289, (2002).

\bibitem{Moser11}
R. Moser and D. Scheder.
\textit{A Full Derandomization of Sch\"{o}ning’s k-SAT Algorithm.} Proceedings of the 44th Annual ACM Symposium on Theory of Computing, (2011).

\bibitem{Schoning99}
U. Sch\"{o}ning.
\textit{A Probabilistic Algorithm for $k$-SAT and Constraint Satisfaction Problems.} Proceedings of the 40th Annual Symposium on Foundations of Computer Science, (1999).

\end{thebibliography}

\end{document}

