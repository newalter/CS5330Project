\documentclass{article}


\usepackage{enumitem}
\usepackage[ruled,vlined,linesnumbered]{algorithm2e}
\usepackage{amsmath}

\title{Applying Sch\"{o}ning's Algorithm on the k-Dominating Set problem}
\author{Li Zeyong, Eldon Chung}
\date{20-Apr-2018}

\begin{document}

\maketitle

\section{Introduction}
	Sch\"{o}ning's algorithm is a randomized algorithm used to solve k-SAT in time $\mathcal{O}^*(\frac{k}{2(k-1)})$, by performing a somewhat randomized search around a randomly chosen initial assignment by means of satisfying clauses. The key concept is that hopefully by doing this we are able to approach the actual witness to the problem instance with some probability. In this paper extension we attempt to apply the same technique and analysis to the NP-Hard problem known as k-Dominating Set, and as a result obtain an even faster solution \par 
	
\section{Problem Definition}
	Given some set $G = (V, E)$, where $V$ is the set of vertices and $E$ the set of $E$, a set $D \subseteq G$ is a dominating set when $\forall v \in V$, we have that either $v$ is in $D$ or it is adjacent to some vertex in $D$. A $k$-Dominating set is when the dominating set is of size at most $k$. The $k$-dominating set problem is a decision problem defined as follows: "Given a graph $G$ and an integer $k$, does there exist a dominating set $D$ of size $k$?\par

\section{Naive Reduction to SAT}
There is a naive reduction from k-Dominating Set to SAT. Consider an instance of a graph $G = (V,E)$ for k-Dominating Set where $|V| = n$, we can construct a SAT formula in the following manner. For each vertex $v_i \in V$, we introduce a variable $x_i$ and assuming that vertex $v_{i_1}, v_{i_2},..., v_{i_j}$ are connected to $v_i$, we construct a clause $C_i = (x_i, x_{i_1}, ..., x_{i_j})$. Lastly, we introduce an addition circuit and a comparison circuit to ensure that $\sum_{i = 1}^n x_i 
\leq k$ \par
Notice that transforming the addition and comparison circuits into a Conjunctive Normal Form inevitably increases the total number of variables to be more than $n$, applying the reduction to SAT and then apply Sch\"{o}ning's Algorithm on the SAT instance might not even beat the naive $O^*({n \choose k})$ search algorithm. Hence, we look into applying the Random Walk directly without go through the reduction to SAT.

\section{Applying the Random Walk Directly}
	\begin{algorithm}[H]
	\caption{Dominating set-Random-Walk}
	\DontPrintSemicolon
	\SetKwInOut{Input}{Input}\SetKwInOut{Output}{Output}
	\Input{$(G)$}
	\Output{A dominating set $D$ or NO if no such set is found}
	\BlankLine
	Randomly select a $k$ sized vertex subset as $D$.\\
	\For{steps = 1 to 3n}{
		\lIf{$D$ is a dominating set }{return $D$}
		Randomly select a vertex $u$ in the current set $D$.\\
		Arbitrarily select a neighbourhood in the graph not covered by $D$\\
		Randomly select vertex $v$ from the neighbourhood of vertices.\\
		Add $v$ to $D$ and remove $u$ from $D$.\\		
	}
	\lIf{$D$ is a dominating set}{return $D$}
	\hspace{7em}\lElse{return NO}
\end{algorithm}
	
\section{Definition of Distance}
	Similar to Sch\"{o}ning's algorithm, we need a notion of distance. We let $D^*$ be some set of vertices that is actually a k-dominating set, and let $D$ be some set of of vertices, both of size $k$. Then we define the distance $r$ between the two to simply be $k - |D \cap D^*|$, or in some sense the number of vertices in which they disagree.\par
\section{Running Time Analysis}
The probability that we go one step closer to the dominating set $D^*$ is $\geq \frac{1}{kd}$. The probability that we go one step further is $\leq \frac{k-1}{k} \cdot \frac{d-1}{d}$. Otherwise, we stay put. Nonetheless, for the ease of analysis, we apply stochastic dominance in considering the case that we also go one step further when we stay put, that is, we go one step further with probability $\frac{kd-1}{kd}$. Let $p_r$ denote the probability that we reach the dominating set within $3n$ steps, where $n$ is the number of vertices, starting from $r$ steps away from the $D^*$. Using the analysis  for random walk from \cite{Schoning99} directly, we have:
\begin{align*}
p_r &\geq \sum_{i = 0}^r {r+2i \choose i} \cdot \Big(\frac{kd-1}{kd}\Big)^i \cdot \Big(\frac{1}{kd}\Big)^{i+r} \\
&\geq \Big(\frac{1}{kd-1}\Big)^r
\end{align*}
There are in total ${n \choose k}$ possible sets. Among them, there are ${k \choose r}{n-k \choose r}$ sets that are exactly $r$ steps from the dominating set $D^*$. Hence, one iteration of the random walk algorithm succeeds with probability:
\begin{align*}
\Pr(\text{success}) &\geq \sum_{r = 0}^{k} \frac{{k \choose r}{n-k \choose r}}{{n \choose k}} \cdot \Big(\frac{1}{kd - 1}\Big)^r \\
&\geq \frac{1}{{n \choose k}} \sum_{r = 0}^k {k \choose r}{n-k \choose r} \cdot \Big(\frac{1}{kd - 1}\Big)^r \\
&\geq \frac{1}{{n \choose k}} \sum_{r = 0}^k {k \choose r}\Big(\frac{n-k}{r}\Big)^r \cdot \Big(\frac{1}{kd-1}\Big)^r \\
&\geq \frac{1}{{n \choose k}} \sum_{r = 0}^k {k \choose r}\Big(\frac{n-k}{k}\Big)^r \cdot \Big(\frac{1}{kd-1}\Big)^r \\
&\geq \frac{1}{{n \choose k}} \sum_{r = 0}^k {k \choose r} \Big(\frac{n-k}{k^2d - k}\Big)^r \cdot 1^{k-r} \\
&\geq \frac{1}{{n \choose k}} \Big(1 + \frac{n-k}{k^2d - k}\Big)^k
\end{align*}
Now we can amplify the probability by repeating and the over running time is:
\begin{align*}
T(n,k) = \mathcal{O}^*\left({n \choose k}\frac{1}{\Big(1 + \frac{n-k}{k^2d - k}\Big)^k}\right) < \mathcal{O}^*\left({n \choose k}\right)
\end{align*}
Therefore, we see that this algorithm runs faster than the naive search in all scenarios. We then try to further approximate this to get a more straightforward running time. Let $H$ be the binary entropy function. \par
\begin{align*}
{n \choose k}\frac{1}{\Big(1 + \frac{n-k}{k^2d - k}\Big)^k} &\leq 2^{nH(k/n)} \Big(\frac{k^2d - k}{k^2d - 2k + n}\Big)^k 
\end{align*}
Finally, by a numerical analysis on the above bound, we found that for $k \leq 0.1404n$, our algorithm runs faster than the currently fastest algorithm that runs in $\mathcal{O}^*(1.5048^n)$\cite{Rooij}.

\section{Discussion and Future Works}
	\subsection{Discussion}
	During our analysis of the running time, it appeared that the degree $d$ did not affect the bound on the running time. Whereas as shown, the value of $k$ relative to $n$ played a huge role.\par
	
	This also implies that if $k$ were to grow sublinearly to $n$, we would obtain a faster algorithm than the current fastest algorithm.\par
	\subsection{Future Works}
	It can be seen that the bound obtained was done so quite loosely, due to the approximations on the summation obtained, as well as due to the application of stochastic dominance when modelling the random walk. Therefore one possible future extension could be to just find a better method of bounding the probabilities.\par
	
\section{Conclusion}	
	To restate our work, we applied a similar concept to Sch\"{o}ning's algorithm as well as technique on bounding and analysis, and managed to obtain a faster solution than the current best for all values of $k$ up to a ratio of 0.1404, as well as cases where $k$ grows sublinearly in n.\par


\begin{thebibliography}{9}
	\bibitem{Rooij}
	van Rooij, J. M. M.; Nederlof, J.; van Dijk, T. C. \textit{Inclusion/Exclusion Meets Measure and Conquer: Exact Algorithms for Counting Dominating Sets}
	Proc. 17th Annual European Symposium on Algorithms, ESA (2009).
	
	
	\bibitem{Schoning99}
	U. Sch\"{o}ning.
	\textit{A Probabilistic Algorithm for $k$-SAT and Constraint Satisfaction Problems.} Proceedings of the 40th Annual Symposium on Foundations of Computer Science, (1999).
\end{thebibliography}
\end{document}

